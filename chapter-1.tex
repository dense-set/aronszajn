\chapter{[the lone chapter]}

\begin{conv}
	Unless stated otherwise, assume the following:\myMargin{
		Filler text\ldots
	}
	\begin{assmplist}
		\item $K\in\{\mathbb R, \mathbb C\}$.
		
		\item $X$ will denote a generic set.
		
		\item Vector spaces will be over $K$.
		
		\item Evaluation functions on any subset of $K^X$ will be denoted by $\delta_x$'s. These are clearly linear maps in case the domain is a subspace of $K^X$.
		
		\item $\mathscr H$ will denote a Hilbert space over $K$.
		
		\item Abusing the notation slightly, the same notation will be used to denote the restriction to $\mathbb{R\to R}$ of $\Re$, $\Im$ and complex conjugation.
		
		\item Whenever $\iprd{\cdot, \cdot}$ is semi-inner-product on a vector space,\footnote{
			That is, it's almost a norm except not possibly satisfying the positive definiteness.
		} we'll use the usual $\norm\cdot$ to denote the induced seminorm.\footnote{Note that a semi-inner-product obeys the Cauchy-Schwarz inequality (just follow Schwarz's proof using the quadratic polynomial).}
		
		\item For any function $k\colon X\times X\to K$, we'll use $k_x$ to stand for $k(\cdot, x)\colon X\to K$
	\end{assmplist}
\end{conv}

\section{Basics}

	\begin{dfn}[Positive semi-definite kernel]
		A p.s.d.\ kernel on a set $X$ is a function $k\colon X\times X\to K$ that is
		\begin{mylist}
			\item \emph{conjugate symmetric}, \ie, $k(y, x) = \overline{k(x, y)}$; and,
			
			\item \emph{positive semi-definite}, \ie, for any $x_1, \ldots, x_n\in X$ and any $\alpha_1, \ldots, \alpha_n\in K$, we have that
			\[
			\sum_{i, j = 1}^n \overline{\alpha_i}\, k(x_i, x_j)\, \alpha_j\ge 0\text.
			\]
		\end{mylist}
	\end{dfn}
	
	\begin{dfn}[Reproducing kernel and RKHS's]
		A \emph{reproducing kernel} for $\mathscr H\subseteq K^X$ is a function $k\colon X\times X\to K$ such that for any $x\in X$, we have that $k_x\in\mathscr H$ and that it obeys the \emph{reproducing property}, \ie,
		\[
		f(x) = \iprd{f, k_x}
		\]
		for any $f\in\mathscr H$.
		
		Further, if $\mathscr H$ admits a reproducing kernel, it's called a \emph{reproducing kernel Hilbert space}.
	\end{dfn}
	
	
	\begin{lem}\label{LEM: repr kernel is a psd kernel}
		A reproducing kernel is a p.s.d.\ kernel.
	\end{lem}
	
	\begin{proof}
		Let $k$ be a reproducing kernel of $\mathscr H\subseteq K^X$.
		\begin{prooflist}
			\item Conjugate symmetry: $k(y, x) = k_x(y) = \iprd{k_x, k_y} = \overline{\iprd{k_y, k_x}} = \overline{k_y(x)} = \overline{k(x, y)}$.
			
			\item Positive semi-definite: Let $x_1, \ldots, x_n\in X$ and $\alpha_1, \cdots, \alpha_n\in K$. Then setting $k_x := k(\cdot, x)$, we have
			\begin{align*}
				\sum\nolimits_{i, j} \overline{\alpha_i}\, k(x_i, x_j)\, \alpha_j 
				& = \sum\nolimits_{i, j} \overline{\alpha_i}\, \iprd{ k_{x_j}, k_{x_i}\, } \alpha_j\\
				& = \sum\nolimits_{i, j} \iprd{\alpha_j k_{x_j}, \alpha_i k_{x_i}}\\
				& = \left\Vert \sum\nolimits_i \alpha_i k_{x_i} \right\Vert^2\\
				& \ge 0\text.\qedhere
			\end{align*}
		\end{prooflist}
	\end{proof}
	
	\begin{rmk}
		The converse is the content of \myRef{THM: Moore-Aronszajn}.
	\end{rmk}
	
	
	\begin{lem}[Characterizing RKHS's]
		$\mathscr H\subseteq K^X$ is an RKHS $\iff$ evaluation functionals on it are continuous.
	\end{lem}
	
	\begin{proof}
		``$\Rightarrow$'': Let $k$ be a reproducing kernel of $\mathscr H$. Then $\abs{\delta_x(f)} = \abs{f(x)} = \abs{\iprd{f, k_x}}\le \norm{k_x}\norm f$ (note that $k_x\in \mathscr H$). Thus $\norm{\delta_x}\le \norm{k_x} < +\infty$.
		
		``$\Leftarrow$'': By Riesz, we can define $k\colon X\times X\to K$ such that $\delta_x = \iprd{\cdot, k_x}$ (since $\delta_x$'s are continuous). Then $f(x) = \delta_x(f) = \iprd{f, k_x}$.
	\end{proof}
	
	
	
	\begin{cor}\label{COR: conv in RKHS implies pointwise convergence}
		Convergence in an RKHS $\implies$ pointwise convergence.\myMargin{
			What about the converse?
		}
	\end{cor}
	
	
	\begin{thm}[RKHS $\mapsto$ r.k. is injective]\label{THM: RKHS --> rk is injective}
		\leavevmode
		\begin{mylist}
			\item An RKHS's reproducing kernel is unique.
			
			\item Conversely, a reproducing kernel makes at most one Hilbert space into an RKHS.
		\end{mylist}
	\end{thm}
	
	\begin{proof}
		\begin{mylist}
			\item Suppose $k$ and $k'$ are reproducing kernels for $\mathscr H\subseteq K^X$. Then for any $f\in \mathscr H$ and $x\in X$, we have $\iprd{f, k_x - k'_x} = f(x) - f(x) = 0$ so that $k_x = k'_x$. Since $x$ was arbitrary, $k = k'$.
			
			\item Let $k$ be a reproducing kernel for $\mathscr H\subseteq K^X$. It suffices to show that $\mathscr H$ is uniquely determined (along with its inner product\footnote{
				Note that addition and scalar multiplication are already determined (namely, pointwise) by definition.
			}) by $k$. Let $\mathscr H_0$ be the subspace of $\mathscr H$ spanned by $k_x$'s for $x\in X$. Note that $\mathscr H_0^\perp = \{0\}$ (because of $k$'s reproducing property) so that $\clos{\mathscr H_0} = (\mathscr H_0^\perp)^\perp = \mathscr H$.\footnote{
				A naïve glance suggests that this fixes $\mathscr H$, but it need \emph{not}! We haven't yet reduced the description of $\mathscr H$ to only that of $k$. The closure of $\mathscr H_0$ (which \emph{does} only depend on $k$) is dependent on the norm topology induced from $\mathscr H$, whose description hasn't been reduced yet.
			} Since $\mathscr H_0$ is determined by $k$, it suffices to have that the topology on $\mathscr H$ is also determined by $k$ $\wimpliedby$ the norm on $\mathscr H$ is determined by $k$. Note that this will also imply that the inner product is uniquely determined (due to polarization).
			
			It suffices to have that the norm is determined on $\mathscr H_0$ for it is dense in $\mathscr H$. Indeed, for $f = \sum_{i = \alpha_i}^n \alpha_i k_{x_i}$, we have
			\begin{align*}
				\norm{f}^2
				& = \sum_{i, j = 1}^n \alpha_i \overline{\alpha_j} \iprd{k_{x_i}, k_{x_j}}\\
				& = \sum_{i, j = 1}^n \overline{\alpha_j}\, k(x_j, x_i)\, \alpha_i\text.\tag*{\qedhere}
			\end{align*}
		\end{mylist}
	\end{proof}
	
	\begin{rmk}
		\begin{mylist}
			\item Note that completeness of $\mathscr H$ was used in writing $\clos{\mathscr H_0} = (\mathscr H_0^\perp)^\perp$.\myMargin{
				An example to show the necessity of this?
			}
			
			\item The converse is the content of \myRef{THM: Moore-Aronszajn}, which together with the above gives the following correspondence for any set $X$:
			\[
			\begin{tikzcd}
				\{\}\ar[r, leftrightarrow] & \{\}\ar[r, leftrightarrow] & \{\}
			\end{tikzcd}
			\]
		\end{mylist}
	\end{rmk}
	
	
	\begin{rmk} % IT IS in PRPii
		Note that completeness of $\mathscr H$ was of no consequence in the results so far. Thus it might seem that we must generalize the definition of reproducing kernel to include \IPS{s}. However, as \myRef{THM: Moore-Aronszajn} will show, if $k$ is an reproducing kernel for an \IPS, then $k$ will also be an reproducing kernel for an \textbf{?????????}
	\end{rmk}
	
	
	
	
	
\section{Moore-Aronszajn}

	We now ask the converse of \myRef{LEM: repr kernel is a psd kernel}. Given a p.s.d.\ kernel $k$ on $X$, we construct an RKHS $\mathscr H$ whose reproducing kernel is precisely $k$. We do so in the following steps:
	\begin{mylist}
		\item Each $k_x$ must lie in $\mathscr H$. Thus, we are motivated to first define a vector space $\mathscr H_0$ spanned by $k_x$'s.
		
		\item We show that there's a unique inner product on $\mathscr H_0$ \wrt which $k$ has the reproducing property.
		
		\item Finally, we complete $\mathscr H_0$, and verify that it's the required RKHS.
	\end{mylist}


	\begin{lem}\label{LEM: space H_0}
		Let $k$ be a p.s.d.\ kernel. Define $\mathscr H_0$ to be the span in $K^X$ of $k_x$'s for $x\in X$. Then $\mathscr H_0$ admits a unique inner product such that $f(x) = \iprd{f, k_x}$ for any $f\in\mathscr H_0$.
	\end{lem}
	
	\begin{proof}
		We show that
		\[
		\iprd{f, g}
		= \sum\nolimits_{i, j} \overline{\beta_j}\, k(y_j, x_i)\,\alpha_i\addtag\label{EQN: inner prod on H_0}
		\]
		defines an inner product on $\mathscr H_0$ for $f = \sum_{i = 1}^m\alpha_i k_{x_i}$ and $g = \sum_{j = 1}^n \beta_j k_{y_j}$. That it's well-defined follows because
		\[
		\sum\nolimits_j \overline{\beta_j} f(y_j)
		= \sum\nolimits_{i. j} \overline{\beta_j}\, k(y_j, x_i)\,\alpha_i
		= \sum\nolimits_i \overline{g(x_i)} \alpha_i\addtag\label{EQN: inner prod on H_0 well-def}
		\]
		where the second equality follows since \uline{$k$ is conjugate symmetric}.
		It's immediate from \myRef{EQN: inner prod on H_0} that $\iprd{\cdot, \cdot}$ is p.s.d.\ and conjugate symmetric (since \uline{$k$ is a p.s.d. kernel}), and from \myRef{EQN: inner prod on H_0 well-def} that it's bilinear with $\iprd{f, k_x} = f(x)$ for any $x\in X$ (take $g = k_x$). Only positive definiteness remains to be shown:
		\begin{subproof}
			Note that $\iprd{\cdot, \cdot}$ is a semi-inner-product so that it obeys Cauchy-Schwarz and induces a seminorm. Now, let $\norm f = 0$. Then for any $x\in X$, we have $\abs{f(x)} = \abs{\iprd{f, k_x}} \le \norm f\norm{k_x} = 0$.
		\end{subproof}
	\end{proof}
	
	
	\begin{lem}
		Continuing \myRef{LEM: space H_0}, let $S := \{\text{Cauchy sequences in $\mathscr H_0$}\}$. Then there exists a linear map $\phi\colon S\to K^X$ that maps Cauchy sequences to their pointwise limits, the kernel of which consists precisely of sequences that converge to $0$ in $\mathscr H_0$.
	\end{lem}
	
	\begin{proof}
		First we show that $\phi$ is indeed well-defined:
		\begin{subproof}
			Let $(f_n)$ be Cauchy in $\mathscr H_0$. Then for any $x\in X$, we have $\abs{ f_m(x) - f_n(x) } = \abs{\iprd{f_m - f_n, k_x}}\le \norm{f_n - f_m}\norm{k_x} \which\to 0$ as $m, n\to \infty$ so that $(f_n(x))$ is Cauchy in $K$ and hence \cgt. Thus, pointwise limits of Cauchy sequences in $\mathscr H_0$ do exist, and since $K$ is Hausdorff, these are unique.
		\end{subproof}
		
		Linearity of $\phi$ is easy. We now compute $\ker\phi$. If $f_n\to 0$ in $\mathscr H_0$, then for any $x\in X$, we have $f_n(x) = \iprd{f_n, k_x}\which\to 0$. Conversely, let $(f_n)$ be a Cauchy sequence in $\mathscr H_0$ that converges to $0$ pointwise. We show that it converges to $0$ in $\mathscr H_0$ as well:
		\begin{subproof}
			Let $\epsilon > 0$. Fix an $N$ and write $f_N = \sum_{i = 1}^n \alpha_i k_{x_i}$. Now, 
			\begin{align*}
				\norm{f_n}^2 
				& = \bigl| \iprd{f_n - f_N, f_n} + \iprd{f_N, f_n}\bigr| \\
				& \le \bigl| \iprd{f_n - f_N, f_n} \bigr| + \biggl|\sum_{i = 1}^n \overline{\alpha_i} f_n(x_i)\biggr|\\
				& \le \norm{f_n - f_N}\norm{f_n} + \sum_{i = 1}^n \abs{\alpha_i}\abs{f_n(x_i)}
			\end{align*}
			so that taking $N$ large enough ensures that the above is eventually less than any arbitrary $\epsilon > 0$.\qedhere
		\end{subproof}
	\end{proof}
	
	Note that $S/\ker\phi$ is precisely the\footnote{``The'' is up to isometries.} metric completion of $\mathscr H_0$.
	
	
	
	\vspace{2in}
	
	
	\begin{lem}
		Continuing \myRef{LEM: space H_0}, define $\mathscr H$ to be the set of pointwise limits of Cauchy sequences in $\mathscr H_0$. Clearly, this is a vector subspace of $K^X$. There exists an inner product on $\mathscr H$ given by
		\[
		\iprd{f, g} = 
		\]
	\end{lem}
	
	
	
	\begin{thm}[Moore-Aronszajn]\label{THM: Moore-Aronszajn}
		Let $k$ be a p.s.d.\ kernel on $X$. Then there exists a \textbf{unique??} RKHS whose kernel is $k$.
	\end{thm}
	
	